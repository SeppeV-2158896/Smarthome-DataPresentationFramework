%% bare_jrnl.tex
%% V1.4b
%% 2015/08/26
%% by Michael Shell
%% see http://www.michaelshell.org/
%% for current contact information.
%%
%% This is a skeleton file demonstrating the use of IEEEtran.cls
%% (requires IEEEtran.cls version 1.8b or later) with an IEEE
%% journal paper.



% *** Authors should verify (and, if needed, correct) their LaTeX system  ***
% *** with the testflow diagnostic prior to trusting their LaTeX platform ***
% *** with production work. The IEEE's font choices and paper sizes can   ***
% *** trigger bugs that do not appear when using other class files.       ***                          ***
% The testflow support page is at:
% http://www.michaelshell.org/tex/testflow/



\documentclass[journal]{IEEEtran}

\usepackage{ifpdf}
\usepackage[pdftex]{graphicx}


\begin{document}
%
% paper title
% Titles are generally capitalized except for words such as a, an, and, as,
% at, but, by, for, in, nor, of, on, or, the, to and up, which are usually
% not capitalized unless they are the first or last word of the title.
% Linebreaks \\ can be used within to get better formatting as desired.
% Do not put math or special symbols in the title.
\title{Enhancing User Comprehension in Smart Homes: \\ A Layered Data Visualization Approach}
%
%
% author names and IEEE memberships
% note positions of commas and nonbreaking spaces ( ~ ) LaTeX will not break
% a structure at a ~ so this keeps an author's name from being broken across
% two lines.
% use \thanks{} to gain access to the first footnote area
% a separate \thanks must be used for each paragraph as LaTeX2e's \thanks
% was not built to handle multiple paragraphs
%

\author{Seppe~Vandenberk \\
        Student of Shared Faculty of Software Engineering Technology at KU Leuven and Hasselt University \\
        Prof. dr. Gustavo Alberto Rovelo Ruiz (Co-Promotor), Prof. dr. Davy Vanacken (Promotor), De heer Sebe Vanbrabant (Internal Supervisor) % <-this % stops a space
    }

\maketitle

% As a general rule, do not put math, special symbols or citations
% in the abstract or keywords.
\begin{abstract}
The abstract goes here.
\end{abstract}

% Note that keywords are not normally used for peerreview papers.
\begin{IEEEkeywords}
Data visualization, layered visualization, smart homes, web user interface, 
\end{IEEEkeywords}






% For peer review papers, you can put extra information on the cover
% page as needed:
% \ifCLASSOPTIONpeerreview
% \begin{center} \bfseries EDICS Category: 3-BBND \end{center}
% \fi
%
% For peerreview papers, this IEEEtran command inserts a page break and
% creates the second title. It will be ignored for other modes.
\IEEEpeerreviewmaketitle



\section{Introduction}
\IEEEPARstart{T}{he} modern home is changing, driven by the integration of technology and the Internet of Things (IoT). These ``smart homes" come equipped with sensors, devices and appliances that collect data on many different things happening in a household. According to \cite{The2022}, there were 63.1 million smart homes in Europe in 2022, each equipped with at least one smart device before the end of the year. The most popular devices were smart thermostats, smart lights, security cameras and smart plugs. In 2024, 10.6\% of household had at least one smart appliance such as an oven, refrigerator or washing machine, but also mowing robots and smart vacuums were included \cite{2024SmartForecast}. 

All this data that is generated holds the potential to improve our lives. It can automate tasks, optimize energy consumption, enhance security and make houses more efficient and comfortable. These houses can, however, also play a role in the innovation of technologies, such as artificial intelligence (AI). AI has made major progress in many domains, as the data that is generated by these smart homes can help train AI models to make correct and precise predictions because of their constant feedback from sensors \cite{NarsunStudios2023UsingFuture}. But this huge potential hinges on user comprehension. 

These smart homes generate a lot of data that can be overwhelming and difficult to understand. The profiles of people who use smart home devices and their tools differ, so a universal solution will probably be difficult to bring to reality. In this paper, we propose a substantiated approach to address these issues by focusing on visualizing power consumption and production data, as well as predicted and actual appliance states. The proposed approach will include a layered chart visualization, where each layer caters to different user needs and levels of understanding. Higher layers will be more precise but harder to understand, while some lower-level layers provide simplicity and fewer details.

\section{Related works}
TODO: introductie van sectie

\subsection{Data visualization domain}

Despite the many studies that have been conducted in the field of data visualization, there is not a whole lot of existing research specifically on the topic of data visualization techniques for smart homes. In contrary to the IoT field, where Big Data processes play a significant role. \cite{Protopsaltis2020DataChallenges} surveyed visualization tools, methods and techniques for data generated by IoT devices. They found that data in smart buildings brought a high-level summary of utility consumption and helped benchmark the portfolios. Data visualization helped detect anomalies and found their causes through deviation visualization. 

\subsection{Chart types}
There exist multiple chart types within the information visualization domain. Each chart type comes with distinct strengths and weaknesses, making it suitability upon the data characteristics and intended communication goals. Data scientists often use conventional techniques. Smart home data is often time series related, meaning each timestamp has only one value. Scatter plots are often used to plot point clouds in 2D and 3D spaces, but this is less useful for time series and will not often be implemented \cite{Protopsaltis2020DataChallenges}.

Given the ubiquity of time series data, engineers and researchers often fall back on the same techniques. The most common technique is the line chart, where the timestamps are often placed along the horizontal axis and the series values are projected on the vertical axis \cite{Heer2009SizingVisualizations}. \cite{Heer2009SizingVisualizations} mentions that many relatively new techniques are based on line charts, incorporating more data on a smaller surface, such as horizon graphs and offset graphs. 

Another chart type is the box plot. This technique is often used to visualize the interval in which a specific value lies. It also uses whiskers to extend the interval, but where the chances are smaller. Histograms or bar plots are also frequently used techniques to indicate amounts. These techniques can be used to display the comparison of energy consumption versus production of a house, here a clustered bar plot would be used. Here the horizontal axis can be labeled by more data types than only the time \cite{Heer2009SizingVisualizations, Midway2020PrinciplesVisualization, TeamBytebeam2023UnleashingRevealed}. 

\subsubsection{Uncertainty visualization}
Subsubsection text here.


% An example of a floating figure using the graphicx package.
% Note that \label must occur AFTER (or within) \caption.
% For figures, \caption should occur after the \includegraphics.
% Note that IEEEtran v1.7 and later has special internal code that
% is designed to preserve the operation of \label within \caption
% even when the captionsoff option is in effect. However, because
% of issues like this, it may be the safest practice to put all your
% \label just after \caption rather than within \caption{}.
%
% Reminder: the "draftcls" or "draftclsnofoot", not "draft", class
% option should be used if it is desired that the figures are to be
% displayed while in draft mode.
%
%\begin{figure}[!t]
%\centering
%\includegraphics[width=2.5in]{myfigure}
% where an .eps filename suffix will be assumed under latex, 
% and a .pdf suffix will be assumed for pdflatex; or what has been declared
% via \DeclareGraphicsExtensions.
%\caption{Simulation results for the network.}
%\label{fig_sim}
%\end{figure}

% Note that the IEEE typically puts floats only at the top, even when this
% results in a large percentage of a column being occupied by floats.


% An example of a double column floating figure using two subfigures.
% (The subfig.sty package must be loaded for this to work.)
% The subfigure \label commands are set within each subfloat command,
% and the \label for the overall figure must come after \caption.
% \hfil is used as a separator to get equal spacing.
% Watch out that the combined width of all the subfigures on a 
% line do not exceed the text width or a line break will occur.
%
%\begin{figure*}[!t]
%\centering
%\subfloat[Case I]{\includegraphics[width=2.5in]{box}%
%\label{fig_first_case}}
%\hfil
%\subfloat[Case II]{\includegraphics[width=2.5in]{box}%
%\label{fig_second_case}}
%\caption{Simulation results for the network.}
%\label{fig_sim}
%\end{figure*}
%
% Note that often IEEE papers with subfigures do not employ subfigure
% captions (using the optional argument to \subfloat[]), but instead will
% reference/describe all of them (a), (b), etc., within the main caption.
% Be aware that for subfig.sty to generate the (a), (b), etc., subfigure
% labels, the optional argument to \subfloat must be present. If a
% subcaption is not desired, just leave its contents blank,
% e.g., \subfloat[].


% An example of a floating table. Note that, for IEEE style tables, the
% \caption command should come BEFORE the table and, given that table
% captions serve much like titles, are usually capitalized except for words
% such as a, an, and, as, at, but, by, for, in, nor, of, on, or, the, to
% and up, which are usually not capitalized unless they are the first or
% last word of the caption. Table text will default to \footnotesize as
% the IEEE normally uses this smaller font for tables.
% The \label must come after \caption as always.
%
%\begin{table}[!t]
%% increase table row spacing, adjust to taste
%\renewcommand{\arraystretch}{1.3}
% if using array.sty, it might be a good idea to tweak the value of
% \extrarowheight as needed to properly center the text within the cells
%\caption{An Example of a Table}
%\label{table_example}
%\centering
%% Some packages, such as MDW tools, offer better commands for making tables
%% than the plain LaTeX2e tabular which is used here.
%\begin{tabular}{|c||c|}
%\hline
%One & Two\\
%\hline
%Three & Four\\
%\hline
%\end{tabular}
%\end{table}


% Note that the IEEE does not put floats in the very first column
% - or typically anywhere on the first page for that matter. Also,
% in-text middle ("here") positioning is typically not used, but it
% is allowed and encouraged for Computer Society conferences (but
% not Computer Society journals). Most IEEE journals/conferences use
% top floats exclusively. 
% Note that, LaTeX2e, unlike IEEE journals/conferences, places
% footnotes above bottom floats. This can be corrected via the
% \fnbelowfloat command of the stfloats package.




\section{Conclusion}
The conclusion goes here.





% if have a single appendix:
%\appendix[Proof of the Zonklar Equations]
% or
%\appendix  % for no appendix heading
% do not use \section anymore after \appendix, only \section*
% is possibly needed

% use appendices with more than one appendix
% then use \section to start each appendix
% you must declare a \section before using any
% \subsection or using \label (\appendices by itself
% starts a section numbered zero.)
%


\appendices
\section{Proof of the First Zonklar Equation}
Appendix one text goes here.

% you can choose not to have a title for an appendix
% if you want by leaving the argument blank
\section{}
Appendix two text goes here.


% use section* for acknowledgment
\section*{Acknowledgment}


The authors would like to thank...


% Can use something like this to put references on a page
% by themselves when using endfloat and the captionsoff option.
\ifCLASSOPTIONcaptionsoff
  \newpage
\fi



% trigger a \newpage just before the given reference
% number - used to balance the columns on the last page
% adjust value as needed - may need to be readjusted if
% the document is modified later
%\IEEEtriggeratref{8}
% The "triggered" command can be changed if desired:
%\IEEEtriggercmd{\enlargethispage{-5in}}

% references section

\bibliographystyle{ieeetr}
\bibliography{references}



% that's all folks
\end{document}


