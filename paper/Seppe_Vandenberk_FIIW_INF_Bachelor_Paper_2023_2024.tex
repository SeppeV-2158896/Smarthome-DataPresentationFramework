%% bare_jrnl.tex
%% V1.4b
%% 2015/08/26
%% by Michael Shell
%% see http://www.michaelshell.org/
%% for current contact information.
%%
%% This is a skeleton file demonstrating the use of IEEEtran.cls
%% (requires IEEEtran.cls version 1.8b or later) with an IEEE
%% journal paper.



% *** Authors should verify (and, if needed, correct) their LaTeX system  ***
% *** with the testflow diagnostic prior to trusting their LaTeX platform ***
% *** with production work. The IEEE's font choices and paper sizes can   ***
% *** trigger bugs that do not appear when using other class files.       ***                          ***
% The testflow support page is at:
% http://www.michaelshell.org/tex/testflow/



\documentclass[journal]{IEEEtran}

\usepackage{ifpdf}
\usepackage[pdftex]{graphicx}


\begin{document}
%
% paper title
% Titles are generally capitalized except for words such as a, an, and, as,
% at, but, by, for, in, nor, of, on, or, the, to and up, which are usually
% not capitalized unless they are the first or last word of the title.
% Linebreaks \\ can be used within to get better formatting as desired.
% Do not put math or special symbols in the title.
\title{Enhancing User Comprehension in Smart Homes: \\ A Layered Data Visualization Approach}
%
%
% author names and IEEE memberships
% note positions of commas and nonbreaking spaces ( ~ ) LaTeX will not break
% a structure at a ~ so this keeps an author's name from being broken across
% two lines.
% use \thanks{} to gain access to the first footnote area
% a separate \thanks must be used for each paragraph as LaTeX2e's \thanks
% was not built to handle multiple paragraphs
%

\author{Seppe~Vandenberk \\
        Student of Shared Faculty of Software Engineering Technology at KU Leuven and Hasselt University \\
        Prof. dr. Gustavo Rovelo Ruiz (Co-Promotor), Prof. dr. Davy Vanacken (Promotor), De heer Sebe Vanbrabant (Internal Supervisor) % <-this % stops a space
    }

\maketitle

% As a general rule, do not put math, special symbols or citations
% in the abstract or keywords.
\begin{abstract}
The abstract goes here.
\end{abstract}

% Note that keywords are not normally used for peerreview papers.
\begin{IEEEkeywords}
Data visualization, layered visualization, smart homes, web user interface, 
\end{IEEEkeywords}






% For peer review papers, you can put extra information on the cover
% page as needed:
% \ifCLASSOPTIONpeerreview
% \begin{center} \bfseries EDICS Category: 3-BBND \end{center}
% \fi
%
% For peerreview papers, this IEEEtran command inserts a page break and
% creates the second title. It will be ignored for other modes.
\IEEEpeerreviewmaketitle



\section{Introduction}
% Coming down from broad subject to smart homes and there definition
\IEEEPARstart{T}{he} modern home is undergoing a transformation driven by the Internet of Things (IoT) and the increasing demand for energy efficiency~\cite{2023TheNews}. There has been a significant evolution from the concept of domotics, the initial idea of home automation, to the integrated systems of smart homes and smart living~\cite{Solaimani2015WhatLiterature}. Smart home technologies are innovations that are digitally connected, automated, or enhance services to the user~\cite{Urwin2023WhatIn, NarsunStudios2023UsingFuture, Sovacool2020SmartPolicies}. Statistics from Berg Insight illustrate this growing trend, with over 63 million smart homes in Europe in 2022, each having at least one smart device by year's end. Popular devices include smart thermostats, lights, security cameras, and smart plugs~\cite{The2022}. By 2024, the focus expanded beyond the convenience of controlling various home devices and systems remotely, automating routine tasks, and streamlining daily activities to save time and effort. In 2024, over 10\% of households adopted at least one smart appliance, including ovens, refrigerators, washing machines, mowing robots, and smart vacuums~\cite{2024SmartForecast}.

% Defining the problem in smart homes
While smart homes offer many benefits, from automation to improved security, a key challenge lies in understanding the true potential of the data they generate~\cite{NarsunStudios2023UsingFuture}. This data can optimize energy usage~\cite{Al-Ali2017AApproach} and personalize user experiences~\cite{Matsui2018AnSensors}. It can even contribute to big data and machine learning-based smart home systems by providing data to train models and validate results with sensor data~\cite{Machorro-Cano2020HEMS-IoT:Saving}. However, while data is very helpful for (automated) applications, its sheer volume and complexity can overwhelm inhabitants.

% Defining the goals of this research and why this research is useful
This paper focuses on the data visualization domain, where technology's evolution has unlocked new possibilities, allowing for more interactive and insightful representations~\cite{Unwin2020WhyVisualization}. However, the full potential of these techniques remains largely untapped in the context of smart homes. There is a lack of user-centred visualization, therefore hindering the full capability of smart home data. Despite data visualization being one of the most appropriate ways to bring information to the user, a "one-size-fits-all" approach is unlikely to be effective, according to a Harvard study~\cite{Unwin2020WhyVisualization}. Therefore, a different approach is needed since smart home users come from various backgrounds and have different technical skill sets.

% Defining how this research is going to apprehend the problem
To address these issues, this research focuses on developing user-friendly data visualizations for smart homes, mainly time series-related data. By leveraging user-centred design principles, the aim is to create layered data charts that enhance user comprehension and interaction with smart home data. Each layer would add extra detail to the chart. This approach ensures that users can customize their experience, making complex data more accessible and actionable. By implementing this layered data visualization strategy, the research seeks to bridge the gap between the vast potential of smart home data and the practical needs of diverse users, ultimately enhancing the usability and effectiveness of smart home systems. To achieve this, several sub-questions are explored:

% Define the subquestions to explain how I designed the framework and why
\begin{itemize}
\item Current practices: How are data visualization techniques currently used in smart home interfaces?
\item User challenges: What challenges do users face when trying to understand these data visualizations?
\item Framework design: How can a framework be designed for layered charts that empower (AI) developers to efficiently present data without requiring in-depth knowledge of data visualization libraries? The framework should include predefined templates and components that handle common visualization tasks, such as displaying time series data and representing uncertainty in AI predictions. It should offer intuitive tools for customizing charts to fit specific needs, ensuring that developers can focus on the AI aspects without being bogged down by the complexities of visualization design.
\end{itemize}

% Give the user a more detailed overview from what view point I start and what I'm trying to accomplish
This research paper contributes to a larger effort to create user-friendly smart home interfaces by answering these questions. It starts by discovering different visualization techniques that users need to understand the presented data better. Following this, the framework will be designed with some of the most used approaches so these can be evaluated on their capability to display uncertainty and their effectiveness in providing sufficient detail for different user profiles

\section{Related work} % This section introduces some interesting research that I will use in my approach to design an efficient framework with the most useful techniques.

% Explain the need of related work
In the field of data visualization, many studies have already been conducted on data visualization in smart homes and its effect on user comprehension. This section reviews these studies and is focused on finding gaps in the research and useful information for the development of the framework. Because there has been extensive research into various data visualization techniques, this section only gives a brief insight into smart home-related research and closely related topics. 

\subsection{General data visualization}
In 2006, Few S. did extensive research~\cite{StephenFew2006InformationDesign} on how to visualize different data series based on different theories, such as user-centred, optimal space use, and optimal use of charts and data types. Many of these results are reflected in current-day studies such as~\cite{Heer2009SizingVisualizations, CastelliWhatVisualization}. Another example is the study by Protopsaltis et al.~\cite{Protopsaltis2020DataChallenges}, which discussed challenges and IoT data visualisation techniques. They compared different chart types based on their strengths and weaknesses and suitability for different communication goals. The research showed that scatter plots, which are frequently used for 2D point clouds, can be used for time series. Since line charts are significantly clearer to the reader, scatter plots are not frequently implemented for time series. However, some newer chart types have proven their role in time series visualization, such as the horizon and offset graphs, but these are also based on a normal line chart. 

Another chart type that is frequently used to visualize time-related data is the column or bar chart. The section from Virtual Event report~\cite{Protopsaltis2020DataChallenges} acknowledges their role in this subject but also states that they can only be used for limited time periods. This means that a continuous data set with many data points can better be shown through a line plot. On the other hand, a column chart can better perform a summary of time periods or a chart requiring a relatively high amount of series (e.g., the total energy consumption from every kitchen appliance in a day).

\subsection{Uncertainty visualization}

Both data predictions and measurements come with a degree of uncertainty~\cite{Padilla2021UncertainVisualizations}. A large amount of research in techniques to draw these uncertainties has been conducted, but the majority focuses on displaying the uncertainty around one single measurement. For the line plots, techniques often fall back on displaying ranges by some kind of border such as other line charts or filled ribbons~\cite{Kamal2021RecentSurvey, Padilla2021UncertainVisualizations, GrietheVisualizingMaking}.

\subsection{Smart home data visualization}

Despite the many studies that have been conducted in the field of data visualization, the amount of research on smart home data is limited. Castelli et al.~\cite{CastelliWhatVisualization} studied a similar subject and developed an end-user-centered smart home interface. They used 5 main predefined widgets, each dedicated to a specific topic: news, security, temperature, energy, and diary. The diary widget displayed any events that happened at home, and the news widget displayed the latest news based on the user's preferences. The security widget was similar to the diary widget, but only displayed the latest home-security-related events. The temperature and energy widgets, however, involve visualizing time-related data. The resulting framework from this paper aims to visualize similar data. The user interface they designed used histograms and line charts to display time-related data series. Heer et al.~\cite{Heer2009SizingVisualizations} also claim that line charts best display time series. Their research furthermore emphasizes the trade-off between data density and accuracy in visualizations: packing more information onto a smaller chart can lead to increased estimation errors. Consequently, finding the right balance between including necessary details and maintaining clarity becomes crucial for effective visualization design. 

\subsection{Layered visualization techniques}

Layered visualizations present information in multiple layers, catering to users with varying needs and technical backgrounds. While research specifically on smart home data visualization is limited, Heer et al.~\cite{Heer2009SizingVisualizations} explored the effectiveness of layered charts for time series data, a common format in smart homes. Although their findings were based on a maximum of 2 sets at a time, the paper suggests that layering can improve user comprehension, especially for complex datasets. Kong M. and Agrawala N.~\cite{Kong2012GraphicalReading} concluded that adding different layers to a graph with each layer offering a different level of detail helps to "support the perceptual and cognitive processes and facilitate chart reading". Their findings suggest that layered visualizations can improve user comprehension compared to traditional single-layer visualizations, especially for data sets with high dimensionality. But it needs to be kept in mind that each layer needs to be tailored to the specific needs of the users.

\section{Methodology} % Explain some results from other researches that had an impact on my design choices. 

So, for this paper, not all of these findings are usable, but techniques such as whiskers on a column chart and density plots can be used on the low periodical series.

\section{Framework architecture} % Explain the design of my framework. And show the user how the interface looks for the developer and end-user. Exhaustive API documentation will be documented in the appendix.


\section{Discussion} % Reflect on the created designs with images, analyse each graph's strengths and weaknesses, and explain future additions that can be helpful for the developers and end-users.

\section{Conclusion}
The conclusion goes here.


% use section* for acknowledgment
\section*{Acknowledgment}


The authors would like to thank...




\bibliographystyle{ieeetr}
\bibliography{references}

\end{document}


